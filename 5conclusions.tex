\chapter{Conclusion}\label{conclusion}

The project brought Vehicle's Python bindings closer to serving as a single artefact for both verification and differentiable training. The refreshed modules ingest the current compiler output for the benchmark set, generate backend-neutral loss programs, and execute them in TensorFlow and PyTorch for the covered scenarios, while acknowledging that long-running demonstrations and a handful of compiler edge cases remain open. This chapter distils the inferences, enumerates the concrete contributions, and outlines the research steps still required.

\section{Conclusion and Discussion}

\begin{enumerate}
	\item The research question is answered for the evaluated benchmarks: specifications compiled with \texttt{vehicle --json compile loss} run unchanged in both verification pipelines and TensorFlow/PyTorch training loops (Sections~\ref{analysis}--\ref{approach}). The new bindings remove the previously observed drift between the compiler and Python mirror for these cases, while two bounds-only fixtures remain deliberately disabled until the upstream compiler bug is resolved.
	\item Backend parity is demonstrated by instantiating the shared \inlinecode{vehicle\_lang.loss} translation with TensorFlow and PyTorch builtins (Section~\ref{analysis}). Pytest evidence shows that the two backends expose matching declaration contexts across the golden specifications, indicating that framework choice no longer forces specification rewrites.
	\item Optional-dependency isolation now matches downstream needs: verification tasks import only the lightweight CLI helpers, while training workflows opt into TensorFlow/PyTorch extras on demand. This separation addresses the installation pain reported at project start and clarifies the boundary between proof-oriented and learning-oriented tooling, albeit without repackaging the legacy modules.
	\item Generated losses are usable beyond toy cases, but only within short-running experiments. Constraint-only and mixed-objective tests (§\ref{analysis}) show that DL2-based losses and FGSM-style samplers integrate with standard optimisers, while the longer training demonstrations remain disabled pending further performance work.
\end{enumerate}

\section{Summary of Contributions}

\begin{enumerate}
	\item Delivered \inlinecode{vehicle\_lang.loss.\_ast}, an internal AST decoder that mirrors the current compiler JSON (including \inlinecode{SearchRatTensor}) and isolates future schema evolution from the public API.
	\item Consolidated the abstraction layer inside \inlinecode{vehicle\_lang.loss.\_abc}, unifying type variables, builtin contracts, sampler interfaces, and translation scaffolding so that each backend implements a single coherent contract.
	\item Refurbished the TensorFlow backend and introduced a PyTorch counterpart that reuse the common Python translation; both are positioned as beta-quality implementations pending further optimisation and larger-scale trials.
	\item Helped design the sampler interface and made FGSM-inspired samplers for both frameworks, enabling quantified searches to yield usable losses during training while allowing users to plug in custom strategies.
	\item Extended the Pytest suite to cover AST synchronisation, backend translation, and short training integration runs, providing regression protection for the abstractions even though CI still skips the longest demonstrations.
\end{enumerate}

\section{Limitations and Future Research}

\begin{enumerate}
	\item Address the two bounds-only quantifier fixtures that remain deliberately skipped in \inlinecode{test\_lossdl2\_load.py} because of the known compiler bug; unblocking them would close the last known gap in AST coverage.
	\item Re-enable the long-running end-to-end demonstrations (bounded TensorFlow training and MNIST robustness) under a dedicated CI job or performance budget. Their reintroduction would provide empirical validation for larger models and datasets.
	\item Explore richer sampler strategies (e.g. quasi-random lattices, certified adversaries, or gradient-free search) that plug into \inlinecode{ABCSampler} while preserving differentiability. Comparative studies could quantify how sampler choice affects convergence and robustness.
	\item Generalise the backend abstractions to additional targets, such as JAX or ONNX-compatible runtimes, to further reduce the effort required to bring Vehicle constraints into diverse ML stacks.
	\item Investigate tighter integration with property-driven training frameworks (e.g. automated curriculum generation or joint optimisation with task loss schedulers), turning the compiled Vehicle losses into first-class citizens inside experiment management tooling.
\end{enumerate}