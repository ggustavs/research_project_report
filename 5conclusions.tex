\chapter{Conclusion}\label{conclusion}

You generally cover three things in the Conclusions section, and each of these usually merits a separate subsection:

\begin{enumerate}
        \item Conclusion \& Discussion
	\item Summary of Contributions
	\item Limitations and Future Research
\end{enumerate}

Conclusions are not a rambling summary of the thesis: they are short, concise statements of the inferences that you have made because of your work. Ideally, they are split up: a super short wrap-up in the heading of the chapter, and then a "Discussion" that reflects the outcomes and relates them to the related work. It helps to organize these as short, numbered paragraphs, ordered from most to least important. All conclusions should be directly related to the research question stated in the beginning. Examples:

\begin{enumerate}
	\item The problem stated in Section 1 has been solved: as shown in Sections ? to ??, an algorithm capable of handling large-scale Zylon problems in a reasonable time has been developed.
	\item The principal mechanism needed in the improved Zylon algorithm is the Grooty mechanism.
	\item Etc.
\end{enumerate}


The Summary of Contributions will be much sought after and carefully read by the examiners. Here you list the contributions of new knowledge that your thesis makes. Of course, the thesis itself must substantiate any claims made here. There is often some overlap with the Conclusions, but this is okay. Concise numbered paragraphs are again best. Organize from most to least important. Examples:

\begin{enumerate}
	\item Developed a much quicker algorithm for large-scale Zylon problems.
	\item Demonstrated the first use of the Grooty mechanism for Zylon calculations.
	\item Etc.
\end{enumerate}

The Future Research subsection is included so that researchers, picking up this work in the future, benefit from the ideas you generated while working on the project. Again, concise, numbered paragraphs are usually best. Sometimes this is done before the summary of contributions, to end the thesis on a positive note.

\newpage \paragraph{The thesis document always closes with a Bibliography}

The list of references is closely tied to the review of the state of the art given in chapter 2. Most examiners scan your list of references looking for important works in the field, so make sure they are listed and referred to. You can cite with \cite{Leunen:Scholars:1992, Zuber-Skerritt:ThesisWriting:1986}, \citep{Zuber-Skerritt:ThesisWriting:1986}, or \citet{Zuber-Skerritt:ThesisWriting:1986}

All references given must be referred to in the main body of the thesis. Organize the list of references either alphabetically by author surname (preferred), or by order of citation in the thesis. 